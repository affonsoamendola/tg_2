O protótipo demonstra um grande potencial de sucesso, a fácil interface entre o Front-End e código Python executando no Back-End permite uma grande variedade de ações serem realizadas em cima de dados, todas as funções de Astropy estão acessíveis, muitas deles já sendo usadas para ajudar a popular os bancos de dados, obtendo informações relevantes de novos arquivos .FITS.

A capacidade de Flask de servir requests http com código HTML, JavaScript e CSS, permite o uso de bibliotecas de análise e visualização de dados escritas usando essas linguagems, como a biblioteca D3.js, que permite visualização de dados.

O fator limitante para os tipos de análise que podem ser feitas diretamente do browser é a complexidade inerente em escrever uma interface gráfica responsiva usando somente HTML, JS e CSS, no entanto isso não significa que uma interface gráfica responsiva não possa ser feita, somente que a complexidade do código pode aumentar drásticamente se cuidados não forem tomados, elevando também os custos de mantimento dessa parte do código.

Todo software precisa ser mantido, somente escrever um software e deixar ele sem mantedor implica em cientistas possívelmente precisarem de um programa, ele ser o único em existência que supre as nescessidades do cientista, e ele não conseguir ser usado devido a algum problema grave já reportado como bug, mas que nunca foi resolvido, pois o criador original do software não tem mais interesse em manter o software, e um outro mantedor nunca foi encontrado.

Mas a existência de um mantedor não implica num software manter um nivel de funcionamento saudavel, diferentes programadores possivelmente trabalharam em um determinado pedaço de um software, isso gera uma grande importancia para a existência de uma boa documentação dentro e fora do software (Como esta monografia), que explicam as motivações e decisões que foram tomadas durante o desenvolvimento do mesmo.

O desenvolvimento do protótipo em um software completo e usável continua, com uma previsão de conclusão até o começo de 2022.