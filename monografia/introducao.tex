\chapter{Introdução}

Estrelas Be são estrelas do tipo B que mostram linhas de emissão de hidrogênio, muitas vezes com variações no espectro, de forma que o fenômeno de emissão seja transitório.(\citealt{classical_be})

\begin{figure}[!ht]
\begin{center}
\setcaptionmargin{1cm}
\includegraphics[width=0.4 \columnwidth,angle=0]{fig/achernar.png}
\caption[Representação Artística de Achernar]{Representação artística de uma estrela Be, Achernar.} 
\label{achernar_fig}
\end{center}
\end{figure}

A variabilidade de estrelas Be deu luz a alguns projetos de banco de dados criados com o objetivo de armazenar a grande quantidade de dados oriundas de observações desse tipo de estrela, já que múltiplas observações são necessárias, e toda observação pode ser valiosa num contexto futuro.

Sendo dois projetos de destaque os projetos BeSS e BeSOS.

\begin{figure}[!ht]
\begin{center}
\setcaptionmargin{1cm}
\includegraphics[width=1.0 \columnwidth,angle=0]{fig/bess_screenshot.png}
\caption[Screenshot do BeSS]{Captura de tela da interface do web-site do projeto BeSS;} 
\label{bess_screenshot_fig}
\end{center}
\end{figure}

O banco de dados do projeto BeSS (Be Star Survey Database) é o maior banco de dados desse tipo de estrela atualmente em existência, possuindo mais de 250 mil entradas de espectros, ele é mantido pelo LESIA (Laboratoire d'Études Spatiales et d'Instrumentation en Astrophysique) do Observatório de Paris.

Esse projeto tem muitas relações com grupos de astrônomos amadores, algo importante para observações de estrelas Be, já que são estrelas variáveis, e eventos interessantes podem acontecer a qualquer momento.(\citealt{bess_database})

\begin{figure}[!ht]
\begin{center}
\setcaptionmargin{1cm}
\includegraphics[width=1.0 \columnwidth,angle=0]{fig/besos_screenshot.png}
\caption[Screenshot do BeSOS]{Captura de tela da interface do web-site do projeto BeSOS;} 
\label{besos_screenshot_fig}
\end{center}
\end{figure}

Por outro lado, o projeto BeSOS tem imagens de somente um espectrógrafo especifico, mas possuí excelentes ferramentas de visualização de dados, como plotagem interativa e calculo de alguns valores relevantes a Ciência de estrelas Be.

A criação de um banco de dados moderno, combinando o volume de dados presentes no BeSS com as ferramentas de visualização, análise e cross-matching do BeSOS está atualmente em andamento, e este trabalho é um plano, explicando e definindo todos os passos necessários para a criação de um software similar.(\citealt{besos_database})

A grande maioria da funcionalidade requerida por um software desse tipo já existe livremente disponível na internet, na forma de bibliotecas FOSS (Free and Open-Source Software, software livre e de código aberto), então a maior parte do trabalho necessário consiste em escrever código que permita a interação harmoniosa entre as diferentes bibliotecas e providenciar uma interface de acesso ao usuário.

Usando esses softwares, uma proposta detalhada deste software é feita no Capítulo 3 deste trabalho.
