Bancos de Dados de imagens e dados astronômicos se provaram essenciais no repertório de ferramentas de um astrônomo moderno, bancos como o BeSS (Be Star Spectra Database) são usados diariamente por pesquisadores para armazenar novos dados aguardando análise, ou aplicar novas técnicas e conhecimentos em dados antigos.

Dados de estrelas Be no momento se encontram separados em alguns bancos de dados, o BeSS, o do BeSOS( Be Stars Observation Survey), o do ESO (European Southern Observatory), nosso objetivo é construir um novo banco de dados, tentando unificar os dados de todos esses bancos em um só lugar.

Muitas das análises normalmente aplicadas a dados de estrelas Be, podem ser feitas diretamente do Browser, alguns bancos mencionados possuem websites onde "Queries" SQL podem ser feitas, ou algumas análises simples podem ser realizadas, mas essa opção não está presente em todos os bancos, especificamente faltando para o maior deles, o BeSS, de onde sairá a grande maioria dos dados para alimentar o BeACON Database